% This must be in the first 5 lines to tell arXiv to use pdfLaTeX, which is strongly recommended.
\pdfoutput=1
% In particular, the hyperref package requires pdfLaTeX in order to break URLs across lines.

\documentclass[11pt]{article}

\usepackage{mystyle}

% Remove the "review" option to generate the final version.
\usepackage[review]{acl}

% Standard package includes
\usepackage{times}
\usepackage{latexsym}

% For proper rendering and hyphenation of words containing Latin characters (including in bib files)
\usepackage[T1]{fontenc}
% For Vietnamese characters
% \usepackage[T5]{fontenc}
% See https://www.latex-project.org/help/documentation/encguide.pdf for other character sets

% This assumes your files are encoded as UTF8
\usepackage[utf8]{inputenc}

% This is not strictly necessary, and may be commented out,
% but it will improve the layout of the manuscript,
% and will typically save some space.
\usepackage{microtype}

% If the title and author information does not fit in the area allocated, uncomment the following
%
%\setlength\titlebox{<dim>}
%
% and set <dim> to something 5cm or larger.

\title{Scaling Switching Language Models}

% Author information can be set in various styles:
% For several authors from the same institution:
% \author{Author 1 \and ... \and Author n \\
%         Address line \\ ... \\ Address line}
% if the names do not fit well on one line use
%         Author 1 \\ {\bf Author 2} \\ ... \\ {\bf Author n} \\
% For authors from different institutions:
% \author{Author 1 \\ Address line \\  ... \\ Address line
%         \And  ... \And
%         Author n \\ Address line \\ ... \\ Address line}
% To start a seperate ``row'' of authors use \AND, as in
% \author{Author 1 \\ Address line \\  ... \\ Address line
%         \AND
%         Author 2 \\ Address line \\ ... \\ Address line \And
%         Author 3 \\ Address line \\ ... \\ Address line}

\author{Justin Chiu \\
  Cornell Tech / Address line 1 \\
  \texttt{jtc257@cornell.edu} \\\And
  Second Author \\
  Affiliation / Address line 1 \\
  Affiliation / Address line 2 \\
  Affiliation / Address line 3 \\
  \texttt{email@domain} \\}

\begin{document}
\maketitle
\begin{abstract}
The accuracy of simple models with discrete latent dynamics,
such as Hidden Markov Models and Probabilistic Context-Free Grammars,
scales with the size of the respective latent spaces -- up to a point.
Recent work has found that scaling discrete latent variable models comes with diminishing returns on accuracy.
We explore the hypothesis that discrete representations are not suitable for language modeling,
due to language's long-tailed phenomena.
We overcome this shortcoming by combining large-scale discrete dynamics
with slow-moving continuous state representations,
and show that this enables simple models to better capture tail phenomena in language modeling.
\end{abstract}

\section{Introduction}
Why should we work on LVMs?
Historical reasons.
Experiment with different representations.

Recent work in scaling discrete latent variable models has shown that their accuracy scales with size \citep{chiu-rush-2020-scaling,yang-etal-2021-pcfgs}.
However, those gains diminish as scale increases.
Additionally, the computational cost of inference greatly increases with scale.
In this report, we examine the shortcomings of these models in a few case studies,
and whether we can overcome those shortcomings.

\section{Long-Tail Phenomena: Rare Words}
\paragraph{Hypothesis:} HMMs will assign lower probability to rare words than LSTMs.
\paragraph{Experiment:} Train both models and plot num occurrences vs average likelihood.

Num occurrences vs Fit

Num occurrences vs Generalization

%\section*{Acknowledgements}

% Entries for the entire Anthology, followed by custom entries
\bibliography{anthology,custom}
\bibliographystyle{acl_natbib}

\appendix

\section{Example Appendix}
\label{sec:appendix}

This is an appendix.

\end{document}
